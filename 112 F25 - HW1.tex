\documentclass[11pt]{article}    

\usepackage{graphicx}
\usepackage{amsfonts}
\usepackage{amsmath}
\usepackage{amssymb}
\usepackage{amsmath,amscd}
\usepackage{amsthm}
\usepackage[T1]{fontenc}
\usepackage[hidelinks]{hyperref}
\usepackage{lmodern}
\hypersetup{
     colorlinks   = true,
     citecolor    = gray
}                      


\usepackage{xypic}
\usepackage{mathrsfs}
\usepackage{upgreek}
\usepackage{esint}
\usepackage[usenames,dvipsnames,svgnames,table]{xcolor}
\usepackage[none]{hyphenat}
\setcounter{tocdepth}{3}
\usepackage{eucal} 
\usepackage{amsthm}
\usepackage{bbm}
\usepackage{textcomp}
\usepackage{wrapfig}
\renewcommand{\abstractname}{\vspace{-\baselineskip}}




\textwidth=6in
\textheight=9in
\hoffset=-0.375in
\voffset=-0.75in
\newtheorem{theorem}{Theorem}[subsection]
\newtheorem{corollary}{Corollary}[subsection]
\newtheorem{lemma}{Lemma}[subsection]
\newtheorem{remark}{Remark}
\newtheorem{claim}{Claim}
\newtheorem{proposition}{Proposition}[subsection]
\newtheorem{example}{Example}
\newtheorem{conjecture}{Conjecture}
\newtheorem{definition}{Definition}[subsection]
\newtheorem{problem}{Problem}[subsection]
\newtheorem{assumption}{Assumption}[subsection]
\newtheorem{condition}{Condition}[subsection]
\newtheorem{introdefinition}{Definition}
\newtheorem{alternatedefinition}{Alternate Definition}[subsection]


\numberwithin{equation}{section}






\makeatletter
\renewcommand\tableofcontents{%
    \@starttoc{toc}%
}
\makeatother


\newcommand*{\TitleFont}{%
      \usefont{\encodingdefault}{\rmdefault}{b}{n}%
      \fontsize{14}{14}%
      \selectfont}
\newcommand*{\AuthorFont}{%
      \usefont{\encodingdefault}{\rmdefault}{b}{n}%
      \fontsize{12}{12}%
      \selectfont}




\newcommand{\kbar}
{\mathchar'26\mkern-9mu k}

\newcommand{\ebar}
{\mathchar'26\mkern-9mu \theta_0}

\newcommand{\ebarblue}
{\mathchar'26\mkern-9mu \textcolor{blue}{\theta_0}}




\DeclareMathOperator*{\lcm}{lcm}
\DeclareMathOperator*{\R}{\mathbb{R}}
\DeclareMathOperator*{\N}{\mathbb{N}}
\DeclareMathOperator*{\Q}{\mathbb{Q}}
\DeclareMathOperator*{\Z}{\mathbb{Z}}
\DeclareMathOperator*{\E}{\mathbb{E}}
\DeclareMathOperator*{\C}{\mathbb{C}}
\DeclareMathOperator*{\A}{\mathbb{A}}
\DeclareMathOperator*{\expp}{\text{exp}}
\DeclareMathOperator*{\lt}{L_{\bullet}}
\DeclareMathOperator*{\roc}{\rotatebox[origin=c]{180}{c}}
\DeclareMathOperator*{\rok}{\rotatebox[origin=c]{180}{$k$}}
\DeclareMathOperator*{\boldv}{\boldsymbol{v}}
\DeclareMathOperator*{\hbarbrown}{\textcolor{brown}{\hbar}}
\DeclareMathOperator*{\e}{\textcolor{purple}{\theta_0_1}}
\DeclareMathOperator*{\ee}{\textcolor{purple}{\theta_0_2}}
\DeclareMathOperator*{\ered}{\textcolor{red}{--\theta_0_1 \theta_0_2}}
\DeclareMathOperator*{\eblue}{\textcolor{blue}{\theta_0_1 + \theta_0_2}}
\DeclareMathOperator*{\eebrown}{\textcolor{brown}{-- \theta_0}}
\DeclareMathOperator*{\ebrown}{\textcolor{brown}{\theta_0}}
\DeclareMathOperator*{\eebrownlevel}{\textcolor{brown}{\theta_0^2}}
\DeclareMathOperator*{\tealN}{\textit{\textcolor{teal}{N}}}
\DeclareMathOperator*{\bluenoise}{\boldsymbol{\Psi}_{\mathbb{T}}}
\DeclareMathOperator*{\pinknoise}{\boldsymbol{\Psi}_{\mathbb{T}}}
\DeclareMathOperator*{\uniform}{\boldsymbol{\rho}_{\star | \mathbb{T}}}
\DeclareMathOperator*{\vcurrent}{\widehat{\textit{\textbf{v}}}}
\newcommand{\dbar}
{\mathchar'26\mkern-9mu \theta_0}
 \setcounter{section}{0}
 
\begin{document}



\hypersetup{linkcolor=black}

\begin{center} 
\textbf{{MATH 112: \underline{Introduction to Analysis}}}\\
\textbf{{Fall 2025 Semester}}\\
\textbf{{Homework \textcolor{black}{1}: Due \textcolor{black}{Tuesday} \textcolor{black}{September} \textcolor{black}{09}, 10:00am PST.}}\\
 \end{center}

 \noindent \textbf{Instructions}
 \begin{itemize}
 \itemsep0em
 \item Write your full name, ``Homework \textcolor{black}{1}'', and the date at the top of the first page.
 \item Show all work and explain your reasoning.  Write in complete sentences.
 \item Typeset your solutions in LaTeX \textit{or, for HW1 only, you may submit a scanned .pdf}.
 \item Each question has multiple parts. Box your final answers.
 \item \underline{Submit a single .pdf file to Gradescope under the assignment ``Homework \textcolor{black}{1}''.}
 \item \underline{You must use Gradescope to electronically match problems to pages in your .pdf}
 \item Questions? Email me or come to office hours (see Moodle).
 \item You are strongly encouraged to work together!  Just write up your own solutions.\\
 \end{itemize}
 \noindent \textbf{Assignment} (2 Problems: 50 + 50 = 100 points total.) \\
  $ \ $\\
   \noindent $\square$ \textbf{Problem 1}  [\textsf{Logic and $\mathbb{D}$}]. In \textsf{L0}, we encountered the finite set $\mathbb{D} = \{0,1,2,3,4,5,6,7,8,9\}$ of digits in base $10$.  Using $d$ as symbol to denote an element of $\mathbb{D}$, consider the predicates \begin{eqnarray} P(d) &=& \textnormal{``$|d-5| \leq 2$''} \nonumber \\ Q(d) &=& \textnormal{``$|d-2| \leq 5$''}  \nonumber \end{eqnarray}
   
   \noindent where $|x|$ is absolute value.  \textit{Careful: if $x$ is a real number, $x$ is not a set, so the notation ``$|x|$'' refers to ``the absolute value of $x$'' and cannot possibly mean ``the cardinality of $x$''.} Determine the truth value of each of the following propositions.
  \begin{itemize}
  \itemsep0em 
   \item 1.1 [10 points] $P(3) \land (P(2) \lor P(1))$ 
   \item 1.2 [10 points] $\neg \big (P(3) \Rightarrow Q(3) \big )$
   \item 1.3 [10 points] $\exists d \in \mathbb{D} \ P(d)$ 
   \item 1.4 [10 points] $\forall d \in \mathbb{D} \ P(d)$
   \item 1.5 [10 points]  $\forall d \in \mathbb{D} \ \big ( P(d) \Rightarrow Q(d) \big )$.\\
\end{itemize}

 
    \noindent $\square$ \textbf{Problem 2}  [\textsf{Logic and $\mathbb{N}$}] In \textsf{L0}, we encountered the infinite set of natural numbers $\mathbb{N} = \{0,1,2,3,4,5,6,7,8,9,10,11, 12, 13, \ldots\}$.  For each $k \in \mathbb{N}$, consider the truth set $$A_{k} = \{n \in \mathbb{N} \ : \ k \leq n\}$$
    
\noindent To prove $a \in A_k$, you have to verify $k \leq a$.  Prove each of the following propositions.  \begin{itemize}
  \itemsep0em 
  \item 2.1 [10 points] $\exists \ell \in \mathbb{N} \ \ell  \in A_7$
  \item 2.2 [10 points] $\exists \ell \in \mathbb{N} \ 7  \in A_{\ell}$
\item 2.3 [10 points] $\neg \big (| \mathbb{N} \setminus A_{10} | = 9 \big )$. \textit{Careful: if $S$ is a set, $|S|$ is the cardinality of $S$} 
\item 2.4 [10 points] $\forall m \in \mathbb{N} \ 2m+1 \in A_m$
\item 2.5 [10 points] $\forall m \in \mathbb{N} \ \big (m \in A_{112} \Rightarrow m \in A_{111} \big )$\\

 \end{itemize}


    
\noindent $\square$ \textbf{Bonus} [X points] Is $\forall n \in \mathbb{Z}_+ \ \Bigg ( \exists q \in \mathbb{Q} \ \Big( (0< q) \land (q <  \frac{1}{n} ) \Big ) \Bigg )$ true or false? Explain. \end{document}