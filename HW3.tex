\documentclass[11pt,twoside]{amsart}

\usepackage{graphicx, amsfonts, amsmath, amssymb, amscd, amsthm, lmodern, enumitem, parskip} \usepackage[T1]{fontenc} \usepackage[hidelinks]{hyperref}

\usepackage{ mathrsfs, upgreek, esint, eucal, bbm, textcomp, wrapfig} \usepackage[usenames,dvipsnames,svgnames,table]{xcolor} \usepackage[none]{hyphenat} \setcounter{tocdepth}{3}

\textwidth=6in \textheight=9in \hoffset=-0.375in \voffset=-0.75in

\DeclareMathOperator*{\R}{\mathbb{R}} \DeclareMathOperator*{\N}{\mathbb{N}} \DeclareMathOperator*{\Q}{\mathbb{Q}} \DeclareMathOperator*{\Z}{\mathbb{Z}} \DeclareMathOperator*{\E}{\mathbb{E}} \DeclareMathOperator*{\C}{\mathbb{C}} \DeclareMathOperator*{\A}{\mathbb{A}}

\theoremstyle{definition}
\newtheorem{prob}{Problem}
\newtheorem{bprob}{$\square$ Problem}




\title {Micheal Bear, HW3, date} 

\begin{document}
	\maketitle	
  
 \begin{bprob}
   %text text text
  [\textsf{Real Functions}] Define $D \subseteq \R$ by $D = \R \setminus \{-1\} = \{ x \in \R \ : \ x \neq -1 \}.$ Consider the real function $f: D \rightarrow \R$ defined by the formula $$f(x) = \frac{x-2}{x+    1}.$$

  \begin{enumerate}[label= 1.\arabic*, itemsep=0.2cm]
    \item {[10 points]} Prove that $f$ is injective 
     \textit{Hint: proceed from the definition in }\textsf{L4} 
    \item {[10 points]} Prove that $1 \not \in$ range $(f)$.
 \textit{Hint: use "proof by contradiction" from }\textsf{P5}
    \item {[10 points]} Prove that range $(f) = \mathbb{R} \setminus \{1\}$.
    \textit{Hint: proceed from the definition in }\textsf{L4}
    \item {[10 points]} Is $\forall x \in D \ \exists u \in \R \ f(x) \leq u$ true or false? Give a proof.
      \textit{Hint:} \textsf{P4} \textit{\& Desmos}
    \item {[10 points]} Is $\exists u \in \R \ \forall x \in D \ f(x) \leq u$ true or false? Give a proof.
      \textit{Hint:} \textsf{P4} \textit{\& Desmos}
 \end{enumerate} 
\end{bprob}

\emph{solution}
  %solution text
\begin{enumerate}[label= 1.\arabic*), itemsep=0.4cm]
  \item %1.1
    a function $f : X \rightarrow Y$ is injective if $\forall a \in X \ \forall b \in X \ (f(a) = f(b) \Rightarrow a=b)$ \bigskip \\
    Consider $a \in X$ and $b \in X$. 
    Then $f(a) = \frac{a-2}{a+1}$ 
    and $f(b) = \frac{b-2}{b+1}.$ \\
    Suppose $f(a) = f(b)$, 
    it follows that $\frac{a-2}{a+1} = \frac{b-2}{b+1}$.
    if you solve for a you get:
    \begin{align*}
      \frac{a-2}{a+1} &= \frac{b-2}{b+1}\\ 
      (a-2)\cdot(b+1) &= (b-2)\cdot(a+1)\\
        ab + a -2b -2 &= ab + b - 2a -2 \\
     ab - ab + a + 2a &= b + 2b -2 +2 \\
                   3a &= 3b \\      
                    a &= b
\end{align*}
   Since the equality simplifies to a = b the statment 
   $\big( f(a) = f(b) \Rightarrow a = b \big)$ is true, 
     because $True \Rightarrow True$. $\qed$ 
    \newpage
  \item %1.2
  The $range(f) = \{ y \in Y : \exists \ x\in X \ y = f(x)\}$.\\ \medskip 
  We attempt a proof by contradiction:\\
  First assume $1\in range \ (f)$. consider $y\in Y$ let $y=1
  $ (since we are looking for the $x \in X$ where the statement$y=f(x)$ is True). Then evaluate $f(x) = 1$:

  \begin{align*}
         f(x) &= \frac{x-2}{x+1} \\
1 \cdot (x+1) &= (x-2) \\
          x+1 &= x-2 \\
        x-x+1 &= -2 \\ 
            1 &= -2
  \end{align*}
    Since $1 \neq -2$ and there are no solutions to this equation, we have reached a contradiciton. Thus, by proof of contradiciton $1 \notin range \ (f) \qed$
          

  \item %1.3
 %   We already know that $1 \notin range(f)$ from the problem above. Then split $\mathbb{R}$ into two subsets: \\
 %  the subset $R_n$ where $R_n: \forall x \in \mathbb{R} < 1 $ , \\
 %  and the subset $R_p$ where $R_p: \forall x \in \mathbb{R} > 1$.  \medskip \\
%Consider the case of $R_n$ d:
% \begin{align*}
%   f(x < 0) &= \frac{(x<0)-2}{(x<0+1)}\\
%            &= \frac{x<-2}{x<1}\\
%            &= ?
% \end{align*} 
%Consider the case of $R_p$ d:
% \begin{align*}
%   f(x > 0) &= \frac{(x>0)-2}{(x>0+1)}\\
%            &= \frac{x>-2}{x>1}
%            &= ?
% \end{align*}
  We already know that $1 \notin range(f)$ from the problem above. we can split this problem into two cases: (1) the case where $x \in D$ for all $r \in \R r < 1$, and (2) the case where $x \in D$ for all $r \in \R r>1$.\\
  Case (1): let $x \in D$. Assume $r \in R$ such that $r<1$ Then solve for x:
    \begin{align*}
      f(x) = \frac{x-2}{x+1} &= r \\
                       (x-2) &= r \cdot (x+1) \\ 
                       x-2  &= rx + r \\ 
                       -2-1 &= rx-x+r \\ 
                       -3-r &= x(r-1) \\
         \frac{-(r+3)}{r-1} &= x
    \end{align*}
    Since the equation is true when $r < 1$, the statement: for all $r \in \R$ where $r<1$, $r \in range(f)$ holds true. 
    Case(2): let $x \in D$. Assume $r \in R$ such that $r>1$ then sove for x:
    \begin{align*}
      f(x) = \frac{x-2}{x+1} &= r \\
                      x - 2  &= rx +1 \\
                      -3-r &= x(r-1) \\
         \frac{-(r+3)}{r-1}&= x
    \end{align*}
    since the equation holds true when $r>1$, the statement: for all $r \in \R$ where $r>1$, $r \in range(f)$ holds true.
since both cases are true, the whole proposition holds true $\qed$
  \newpage
\item %1.4
    $\forall x \in D \ \exists \ u \in \    R \ f(x) \leq u$ \\
   % Otherwise stated as: For all x in D there exists a u in the set of real numbers such that f(x) $\leq$ u. 
   %Consider $x \in D$ and $u \in D$. 
  % Assume $f(x) \leq u$, and solve for u:
 %
%    \begin{align*}
%     f(x) = \frac{x-2}{x+1} &\leq u \\
%                     (x-2)  &\leq u \cdot (x+1) \\
%                      x - 2 &\leq ux +1 \\ 
%                   x - 2 -1 &\leq ux \\
%                   -3 &\leq ux + x \\
%                   -3 &\leq x(u+1) \\
%                -3x &\leq u +1
%   \end{align*}
  Let $x \in D$. Nominate $u \in \R$ in terms of x by setting $u = x$ then:
    \begin{align*}
      f(x) = \frac{x-2}{x+1}  &\leq u\\
              \frac{x-2}{x+1} &\leq x \\
                         (x-2)&\leq x \cdot (x+1) \\
                         x-2 &\leq x^2 +1 \\
                         -2-1 &\leq x^2 -x \\
                         -3 &\leq x(x-1).
    \end{align*}
    There are two solutions to $-3 \leq x(x-1)$:
    $$-3 \leq x$$
    and 
    $$-3 \leq x-1 $$
     $$-4 \leq x.$$ 

     Since both are acceptable x values the statement: $f(x) \leq u$ holds true $\qed$

  \item %1.5
    $\exists u \in \R \ \forall x \in D \ f(x) \leq u$ \\
   % Otherwise stated as: there exists a u in the set of real numbers where all x in D such that$f(x) \leq u$.
%   To prove that a proposition P is false, it is sufficent to prove that $\neg P$ is true. The negation of our proposition is equivelent to:
%   \begin{align*}
%   \neg\big(\exists u\in\R\ \forall x\in D\ f(x)\leq u\big)
%       &\Leftrightarrow \forall u \in \R \neg (\forall x \in D\ f(x) \leq u) 
%         \\ 
%       &\Leftrightarrow \forall u \in \R \exists x \in D
%           \ \neg(f(x) \geq u )
%         \\
%       &\Leftrightarrow \forall u \in \R \exists x \in D \ f(x) < u 
%   \end{align*}
%hus it suffices to prove for all $u \in \R$ there is some $x\in D$, such that $f(x) < u$.\medskip
%
%let $u \in \R$. Nominate $x \in D$ in terms of u by seting $x = u$, then:
%\begin{align*}
%  f(x) = \frac{x-2}{x+1} &< u \\  
%  f(u) = \frac{u-2}{u+1} &< u
          % nvm i still dont know
%\end{align*}
    let $u \in \R$ Assume $x\in D$ such that $f(x) \leq u$. Solve for x. 
    \begin{align*}
      f(x) = \frac{x-2}{x+1} &\leq u \\
                  (x-2) &\leq u \cdot (x+1) \\
                  x - 2 &\leq ux + 1\\
                  -2 -1 &\leq ux -x \\
                     -3 &\leq - x(u-1)\\
                     \frac{-3}{u-1} &\leq  -x \\
                     \frac{3}{u-1} &> x
    \end{align*}
    since the equation can reverse the statement holds true $\qed$
%not finished
\end{enumerate}

\newpage
    
\begin{bprob}
  {[ Compositions, Implications, and Counterexamples ]} \\ 
Let $X$, $Y$, and $Z$ be three sets (possibly infinite) \\ 
and let $f \ : \ X \rightarrow Y$ and $g \ : Y \rightarrow Z$ be two functions \\
since codomain$(F) = Y =$ domain$(g)$, $g \circ f \ : X \rightarrow Z$ is a well defined function \\
prove that each given implication below is false by providing an explicit counterexample 
\begin{enumerate}[label= 2.\arabic*, itemsep=0.2cm]
  \item {[10 points]} if $f$ is constant and $g$ is bijective then $g \circ f$ is surjective
  \item {[10 points]} if $|X| \leq |Z|$ then $g \circ f$ is injective
  \item {[10 points]} if $g \circ f$ is bijective then $|X| =|Y|$


\end{enumerate}

\end{bprob}
\emph{solution}

\begin{enumerate}[label= 2.\arabic*), itemsep=0.4cm]
  \item %2.1 
    %somehting about mismatch of domain and codomain

 A function  $f:X \to Y$ is surjective if $\forall y \in Y \ \exists x \in X \ y = f(x)$. \medskip \\
Consider $y \in Y$, let $x \in X$, let$z \in Z$. 
The function $g \circ f$ can be rewritten as: $g(f(x))$. 
Since f is constant no matter what x you put in f, you will always get back out the same y. we can re write our function as $g(y)$. Since there is only one input $y \in Y$ if |y| > 1 we can't hit all $z \in Z$ of our codomain. Since g is bijective each input will have exactly one output. This shows that the proposition is False. $\qed$


  \item %2.2
    Just because the domain is less than the codomain does not mean that it will only go to one output. 

    %--> (x^2?)
    
%  the cardinality of the domain and codomain don't tell us that a

  \item %2.3
    Let $ X = \{1,2,3,4\}$ and let \{1,2,3,4,5,6,7,8\}.
   $f(x) = x+2$. A function being injective only guaranties the $range(f)$ has the same cardinality of X. 
    

\end{enumerate}

\newpage
\begin{bprob} {[Compositions, Surjectivity, and Injectivity]} \\
  Let $X$,$Y$, and $Z$ be three sets (possibly infinite) \\
  and let $f \ : \ X \rightarrow Y$ and $g \ : Y \rightarrow Z$ be two functions.

  \begin{enumerate}[label= 3.\arabic*, itemsep=0.2cm]
    \item {[10 points]} Prove that if $f$ is surjective and $g$ is surgective then $g \circ f$ is surjective.
    \item {[10 points]} prove that if $f$ is injective and $g$ is injective then $g \circ f$ is injective

  \end{enumerate}
\end{bprob}
\emph{solution}

\begin{enumerate}[label=3.\arabic*, itemsep = 0.4cm]
  \item %3.1
        a function  $f:X \to Y$ is surjective if $\forall y \in Y \ \exists x \in X \ y = f(x)$ \bigskip \\
 Consider $y \in Y$, let $x \in X$.
 The composition $(g\circ f)(x)$ can be re written as: $g(f(x))$.
Since $f(x)$ is surjective, we know that for all $y \in Y$ there exists an $x \in X$ such that $y = f(x)$. it follows that we can then substitute $f(x)$ for the $y \in Y$ that corisponds with the x. That then gives us $g(y)$. since we know g(y) is surjective. the whole function is surjective $\qed$

  \item %3.2
     a function $f : X \rightarrow Y$ is injective if $\forall a \in X \ \forall b \in X \ (f(a) = f(b) \Rightarrow a=b)$ \bigskip \\
       Consider $a \in X$, let $b \in x$.
The composition $g \circ f$ can be rewritten as $f(g(x))$. since $g(x)$ is injective we know that for all $a \in X$ and for all $b \in X$, $f(a) = f(b)$ implies $a=b$. It follows that if g is surjective, then $g(f(a)) = g(f(b))$ this implies that $a=b$. thus $g \circ f$ is injective $\qed$ 
    
    
\end{enumerate}
	



\end{document}


