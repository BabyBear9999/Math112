\documentclass[11pt]{article}    

\usepackage{graphicx}
\usepackage{amsfonts}
\usepackage{amsmath}
\usepackage{amssymb}
\usepackage{amsmath,amscd}
\usepackage{amsthm}
\usepackage[T1]{fontenc}
\usepackage[hidelinks]{hyperref}
\usepackage{lmodern}
\hypersetup{
     colorlinks   = true,
     citecolor    = gray
}                      


%\usepackage{xypic}
\usepackage{mathrsfs}
\usepackage{upgreek}
\usepackage{esint}
\usepackage[usenames,dvipsnames,svgnames,table]{xcolor}
\usepackage[none]{hyphenat}
\setcounter{tocdepth}{3}
\usepackage{eucal} 
\usepackage{amsthm}
\usepackage{bbm}
\usepackage{textcomp}
\usepackage{wrapfig}
\renewcommand{\abstractname}{\vspace{-\baselineskip}}




\textwidth=6in
\textheight=9in
\hoffset=-0.375in
\voffset=-0.75in
\newtheorem{theorem}{Theorem}[subsection]
\newtheorem{corollary}{Corollary}[subsection]
\newtheorem{lemma}{Lemma}[subsection]
\newtheorem{remark}{Remark}
\newtheorem{claim}{Claim}
\newtheorem{proposition}{Proposition}[subsection]
\newtheorem{example}{Example}
\newtheorem{conjecture}{Conjecture}
\newtheorem{definition}{Definition}[subsection]
\newtheorem{problem}{Problem}[subsection]
\newtheorem{assumption}{Assumption}[subsection]
\newtheorem{condition}{Condition}[subsection]
\newtheorem{introdefinition}{Definition}
\newtheorem{alternatedefinition}{Alternate Definition}[subsection]


\numberwithin{equation}{section}

%my custom stuff
 \newtheorem{bprob}{$\square$ Problem}
 \usepackage{enumitem} 
 \DeclareMathOperator*{\OO}{\mathbb{O}}
 \DeclareMathOperator*{\I}{\mathbb{I}}
 \DeclareMathOperator*{\+}{\oplus}
 \DeclareMathOperator*{\x}{\otimes}


\makeatletter
\renewcommand\tableofcontents{%
    \@starttoc{toc}%
}
\makeatother


\newcommand*{\TitleFont}{%
      \usefont{\encodingdefault}{\rmdefault}{b}{n}%
      \fontsize{14}{14}%
      \selectfont}
\newcommand*{\AuthorFont}{%
      \usefont{\encodingdefault}{\rmdefault}{b}{n}%
      \fontsize{12}{12}%
      \selectfont}




\newcommand{\kbar}
{\mathchar'26\mkern-9mu k}

\newcommand{\ebar}
{\mathchar'26\mkern-9mu \theta_0}

\newcommand{\ebarblue}
{\mathchar'26\mkern-9mu \textcolor{blue}{\theta_0}}




\DeclareMathOperator*{\lcm}{lcm}
\DeclareMathOperator*{\R}{\mathbb{R}}
\DeclareMathOperator*{\N}{\mathbb{N}}
\DeclareMathOperator*{\Q}{\mathbb{Q}}
\DeclareMathOperator*{\Z}{\mathbb{Z}}
\DeclareMathOperator*{\E}{\mathbb{E}}
\DeclareMathOperator*{\C}{\mathbb{C}}
\DeclareMathOperator*{\A}{\mathbb{A}}
\DeclareMathOperator*{\expp}{\text{exp}}
\DeclareMathOperator*{\lt}{L_{\bullet}}
\DeclareMathOperator*{\roc}{\rotatebox[origin=c]{180}{c}}
\DeclareMathOperator*{\rok}{\rotatebox[origin=c]{180}{$k$}}
\DeclareMathOperator*{\boldv}{\boldsymbol{v}}
\DeclareMathOperator*{\hbarbrown}{\textcolor{brown}{\hbar}}
\DeclareMathOperator*{\e}{\textcolor{purple}{\theta_0_1}}
\DeclareMathOperator*{\ee}{\textcolor{purple}{\theta_0_2}}
\DeclareMathOperator*{\ered}{\textcolor{red}{--\theta_0_1 \theta_0_2}}
\DeclareMathOperator*{\eblue}{\textcolor{blue}{\theta_0_1 + \theta_0_2}}
\DeclareMathOperator*{\eebrown}{\textcolor{brown}{-- \theta_0}}
\DeclareMathOperator*{\ebrown}{\textcolor{brown}{\theta_0}}
\DeclareMathOperator*{\eebrownlevel}{\textcolor{brown}{\theta_0^2}}
\DeclareMathOperator*{\tealN}{\textit{\textcolor{teal}{N}}}
\DeclareMathOperator*{\bluenoise}{\boldsymbol{\Psi}_{\mathbb{T}}}
\DeclareMathOperator*{\pinknoise}{\boldsymbol{\Psi}_{\mathbb{T}}}
\DeclareMathOperator*{\uniform}{\boldsymbol{\rho}_{\star | \mathbb{T}}}
\DeclareMathOperator*{\vcurrent}{\widehat{\textit{\textbf{v}}}}
\newcommand{\dbar}
{\mathchar'26\mkern-9mu \theta_0}
 \setcounter{section}{0}
 
\begin{document}



\hypersetup{linkcolor=black}

\begin{center} 
\textbf{{MATH 112: \underline{Introduction to Analysis}}}\\
\textbf{{Fall 2025 Semester}}\\
\textbf{{Homework \textcolor{black}{4}: Due \textcolor{black}{Tuesday} \textcolor{black}{October} \textcolor{black}{07}, 10:00am PST.}}\\
 \end{center}
 %$ \ $\\
 %\noindent \textbf{Instructions}
 %\begin{itemize}
 %  \setlength\parskip{0pt}
 %\itemsep0em
 %\item Write your full name, ``Homework \textcolor{black}{4}'', and the date at the top of the first page.
 %\item Show all work and explain your reasoning.  Write in complete sentences.
 %\item Typeset your solutions in LaTeX.
 % \item Each question has multiple parts. \textcolor{purple}{*\textit{New: restate each subquestion above your work}*}.
 %\item \underline{Submit a single .pdf file to Gradescope under the assignment ``Homework \textcolor{black}{4}''.}
 %\item \underline{You must use Gradescope to electronically match problems to pages in your .pdf}
 %\item Questions? Email me or come to office hours.
 %\item You are strongly encouraged to work together!  Just write up your own solutions.
 %\end{itemize}

 \noindent \textbf{Assignment} (2 Problems: 50 + 50 = 100 points total.)\medskip 
 
   \noindent $\square$ \textbf{Problem 1}  [\textsf{Fields}] Let $(F, \oplus , \mathbb{O}, \otimes, \mathbb{I})$ be a field satisfying the 10 field axioms from \textsf{L7}. \\
   Prove each of the following propositions, indicating which field axioms you use in each step.\begin{itemize}
  \itemsep0em 
        \item 1.1 [10 points] $\forall x \in F \ \forall y \in F \ \forall c \in F \ \Big ( (x \oplus c = y \oplus c) \ \Rightarrow x=y \Big ) $.
   \item 1.2 [10 points] $\forall x \in F \ \mathbb{O} \otimes x = \mathbb{O}$.  \textit{Hint: $\mathbb{O} = \mathbb{O} \oplus \mathbb{O}$.}
      \item 1.3 [10 points] $\forall x \in F \ \forall y \in F \ \Big ( (x = \mathbb{O} \ \lor \ y = \mathbb{O})  \ \Rightarrow \ x \otimes y = \mathbb{O} \Big ) $.
   \item 1.4 [10 points] $\forall x \in F \ \forall y \in F \ (x \otimes y = \mathbb{O}   \ \Rightarrow \ (x = \mathbb{O} \ \lor \ y = \mathbb{O}))$.
   \item 1.5 [10 points] $\neg \Big ( \exists \sigma \in F \ \mathbb{O} \otimes \sigma= \mathbb{I} \Big ) $.
   \end{itemize}
\medskip 
\begin{enumerate}[label= 1.\arabic*), itemsep=0.4cm]
  \item %1.1
    $\forall x \in F \ \forall y \in F \ \forall c \in F \ \Big ( (x \oplus c = y \oplus c) \ \Rightarrow x=y \Big ) $. \\
    Assume $x \oplus c = y \oplus c$. By the hypothetical strategy, if $x=y$ when we make this assumption, the original statement is $True$. Since, by axiom 4, we know that all elements have $\oplus$-inverses. if we $\oplus (-c)$ to each side:
    \begin{align*}
      x \oplus c &= y \oplus c\\
      x \oplus c \oplus (-c) &= y \oplus c \oplus (-c)\\
      x \oplus \mathbb{O} &= y \oplus \mathbb{O}\\
      x  &= y.
    \end{align*}
    Since by axoim 4 any value $\+$ its inverse equals $\OO$ the equation simplifies down to $x \+ \OO = y \+\OO$. And since, by axiom 3, any value plus the identity $\OO$ equals itself we can simplify to $x = y$. Since we have proved that when we assume $x \+ c = y \+ c$, $x=y$ is $True$, by the hypothetical stretegy, $(x \+ c = y \+ c) \Rightarrow x = y)$ is $True$ for all $x,y,c \in F$. $\qed$
      
  \item %1.2
    $\forall x \in F \ \mathbb{O} \otimes x = \mathbb{O}$\\
    let $x \in F$. If we use the Hint that $\OO = \OO \+ \OO$ and substitute $\OO$ for $\OO \+ \OO$ in the original equation: 
    \begin{align*}
      \OO \x x &= (\OO \+ \OO) \x x \\
             &= (\OO \x x)\+ (\OO \x x).%&\text{Step 1}
    \end{align*} 
    Using axiom 9, we can take $(\OO \+ \OO)$ and distribute $\x$ over $\+$ to get $(\OO \x x) \+ (\OO \x x)$.\\
    Then we can use axiom 4 to take the inverse of $(x \+ c)$ and $\+ (-(\OO\+ x))$ to both sides:
    \begin{align*}
      (\OO \x x) \+ (-(\OO \+ x)) &= 
              (\OO \x x)\+(\OO\x x)\+(-(\OO\x x))\\
      \OO &= (\OO \x x)\+ \OO\\
         &= \OO \x x.
    \end{align*}
    Since a value $\+$ its inverse equals $\OO$ by axiom 9, $(\OO \x x ) \+ (-(\OO \x x)) = \OO$ and we can substitute one for the other on both sides. Since, by axiom 3 a value $\+$ the identity $\OO$, it follows that $(\OO \x x) \+ \OO = \OO \x x$. Thus, by the slinky method, $\OO = \OO \x x$. $\qed$


  \item %1.3
    $\forall x \in F \ \forall y \in F \ \Big ( (x = \mathbb{O} \ \lor \ y = \mathbb{O})  \ \Rightarrow \ x \otimes y = \mathbb{O} \Big ) $\\
 First Assume $(x = \OO \ \lor \ y = \OO)$. By the Hypothetical strategy, if we can prove $x \x y = \OO$ under this assumption, the original statement is $True$. if $(x = \OO \ \lor \ y = \OO)$, we can evaluate the implication by thinking of two cases: case (1) $x = \OO$, and case (2) $y = \OO$.\\
 Case (1): $x = \OO$ if we substitute our value of x into the statement $x \x y$:
 \begin{align*}
   x \x y &= \OO \x y \\
          &= \OO.
 \end{align*}
Since we proved that the identity, $\OO$, $\x$, any value equals the identity, $\OO$ in problem 2, it follows that $\OO \x y = \OO$ 

   %1.4
    $\forall x \in F \ \forall y \in F \ (x \otimes y = \mathbb{O}   \ \Rightarrow \ (x = \mathbb{O} \ \lor \ y = \mathbb{O}))$
  \item %1.5
    $\neg \Big ( \exists \sigma \in F \ \mathbb{O} \otimes \sigma= \mathbb{I} \Big ) $
\end{enumerate}

   \newpage
    \noindent $\square$ \textbf{Problem 2}  [\textsf{Induction}] Prove each of the following propositions by induction.
      \begin{itemize} 
   \itemsep0em 
  \item 2.1 [10 points] $ \exists n_0 \in \mathbb{N} \  \forall n \in \mathbb{N} \ \Big (n_0 \leq n \ \Rightarrow \ 2^n < n! \Big )$. \textit{where $n! = \prod\limits_{k=1}^n k$ and $0!=1$}.
 \item 2.2 [10 points] $ \exists n_0 \in \mathbb{N} \ \forall n \in \mathbb{N} \ \Big (n_0 \leq n \ \Rightarrow \frac{1}{n} < 0.112 \Big )$.
 \item 2.3 [10 points] $\exists n_0 \in \mathbb{N} \ \forall n \in \mathbb{N} \ \Big (n_0 \leq n \ \Rightarrow \ 1+0.112n \leq (1+ 0.112)^n \Big )$
  \item 2.4 [10 points] $\exists n_0 \in \mathbb{N} \ \forall n \in \mathbb{N} \ \Big (n_0 \leq n \ \Rightarrow \ \sum\limits_{j=1}^n j= \frac{n(n+1)}{2} \Big )$.
  \item 2.5 [10 points] $\exists n_0 \in \mathbb{N} \ \forall n \in \mathbb{N} \ \Big (n_0 \leq n \ \Rightarrow \ \sum\limits_{k=0}^n z^k = \frac{1-z^{n+1}}{1-z} \Big )$ \textit{for $z \in \mathbb{C}$ and $z \neq 1$}.
   \end{itemize}
  {\small 
  \noindent \textit{Recall from \textnormal{\textsf{L10}}: to prove $\exists n_0 \in \mathbb{N} \ \forall n \in \mathbb{N} \ \Big ( n_0 \leq n \ \Rightarrow \ P(n) \Big )$ by induction requires two steps:
  \indent \textnormal{(i)} nominate a base case $n_0 \in \mathbb{N}$ and prove $P(n_0)$\\
  \indent \textnormal{(ii)} prove $\forall n \in \mathbb{N} \ \Big ( (n_0 \leq n)  \land P(n) \ \Rightarrow \ P(n+1) \Big )$ using the hypothetical strategy from \textnormal{\textsf{P3}}.\\ 
In the inductive step \textnormal{(ii)}, try to format your proof of the implication via the ``slinky method''.}}

 \begin{enumerate}[label= 2.\arabic*), itemsep=0.4cm]
  \item %2.1
    $ \exists n_0 \in \mathbb{N} \  \forall n \in \mathbb{N} \ \Big (n_0 \leq n \ \Rightarrow \ 2^n < n! \Big )$. \textit{where $n! = \prod\limits_{k=1}^n k$ and $0!=1$}.
    \textit{proof.} we will prove this by induction. First note that the statement holds when $n=4$. 


  \item %2.2
    $ \exists n_0 \in \mathbb{N} \ \forall n \in \mathbb{N} \ \Big (n_0 \leq n \ \Rightarrow \frac{1}{n} < 0.112 \Big )$

  \item %2.3
    $\exists n_0 \in \mathbb{N} \ \forall n \in \mathbb{N} \ \Big (n_0 \leq n \ \Rightarrow \ 1+0.112n \leq (1+ 0.112)^n \Big )$

  \item %2.4
    $\exists n_0 \in \mathbb{N} \ \forall n \in \mathbb{N} \ \Big (n_0 \leq n \ \Rightarrow \ \sum\limits_{j=1}^n j= \frac{n(n+1)}{2} \Big )$

  \item %2.5
    $\exists n_0 \in \mathbb{N} \ \forall n \in \mathbb{N} \ \Big (n_0 \leq n \ \Rightarrow \ \sum\limits_{k=0}^n z^k = \frac{1-z^{n+1}}{1-z} \Big )$ \textit{for $z \in \mathbb{C}$ and $z \neq 1$}.

\end{enumerate}   

\noindent $\square$ \textbf{Bonus} [X points] Construct an example of a field $F$ with $|F|=4$.

 \end{document}
