\documentclass[11pt,twoside]{amsart}
\usepackage{amssymb, amsmath, enumerate, palatino, hyperref}
\usepackage{mathrsfs}



\newtheorem{prob}{Problem}

\title {Micheal Bear, Homework 1 , Sep, 4, 2025} 

\begin{document}
	\maketitle	

 \noindent $\square$ \textbf{Problem 1}  [\textsf{Logic and $\mathbb{D}$}]. In \textsf{L0}, we encountered the finite set $\mathbb{D} = \{0,1,2,3,4,5,6,7,8,9\}$ of digits in base $10$.  Using $d$ as symbol to denote an element of $\mathbb{D}$, consider the predicates \begin{eqnarray} P(d) &=& \textnormal{``$|d-5| \leq 2$''} \nonumber \\ Q(d) &=& \textnormal{``$|d-2| \leq 5$''}  \nonumber \end{eqnarray}


\noindent where $|x|$ is absolute value.  \textit{Careful: if $x$ is a real number, $x$ is not a set, so the notation ``$|x|$'' refers to ``the absolute value of $x$'' and cannot possibly mean ``the cardinality of $x$''.} Determine the truth value of each of the following propositions.
  \begin{itemize}
  \itemsep0em 
   \item 1.1 [10 points] $P(3) \land (P(2) \lor P(1))$ 
   \item 1.2 [10 points] $\neg \big (P(3) \Rightarrow Q(3) \big )$
   \item 1.3 [10 points] $\exists d \in \mathbb{D} \ P(d)$ 
   \item 1.4 [10 points] $\forall d \in \mathbb{D} \ P(d)$
   \item 1.5 [10 points]  $\forall d \in \mathbb{D} \ \big ( P(d) \Rightarrow Q(d) \big )$.
\end{itemize}



\emph{solution}

	%solution goes here%
\begin{enumerate}[1)]
	\item
		%sol'n% 
    \begin{align*}
          &P(3) \land (P(2) \lor P(1)) \\
          &= ( |3-5| \leq 2) \land \Big((|2-2| \leq 2) \lor (|1-5| \leq 2) \Big) \\
          &= (2 \leq 2) \land \Big( ( 0 \leq 2) \lor (4 \leq 2) \Big) \\
          &=(True) \land ( True \lor False) \\
          &=True \land (True) \\
          &=True
    \end{align*}
	\item
		%sol'n%
    \begin{align*}
      \neg \big( P(3) \Rightarrow Q(3) \big)\\
      &= \neg \big( (|3-5| \leq 2) \Rightarrow (|3-2| \leq 5) \big)\\
      &= \neg \big( (2 \leq 2) \Rightarrow (1 \leq 5) \big)\\
      &= \neg \big( True \Rightarrow True \big)\\
      &= \neg (True) \\
      &= False
    \end{align*}
	\item 
		%sol'n%
    \begin{align*}
      \exists d \in \mathbb{D} P(d)\\
      &= \exists d \in \mathbb{d} |d-5| \leq 2 \\
    \end{align*}
    you can also write this as: \\
    there exists a number d in the set of digits such that $|d-5| \leq 2$\\
    this statement is true because: \\
   let d = 4
    $$|d-5| = |4-5|$$
    $$1 \leq 2$$
   this makes the stament $True$
	\item           
		%sol'n%  
    %$\forall d \in \mathbb{D} \ P(d)$

    \begin{align*}
      \forall d \in \mathbb{D} \ P(d) \\
      &= \forall d \in \mathbb{D} \ |d-5| \leq 2
    \end{align*}
    let $d = 1$ 
    $$|1-5| = 4$$
    $$4 \leq 2$$
    since $4 > 2$ this statment is $False$
  \item
    \begin{align*}
      \forall d \in \mathbb{D} \ \big ( P(d) \Rightarrow Q(d) \big ) \\
      &= \forall d \in \mathbb{D} \big ( |d - 5| \leq 2 \Rightarrow |d-2| \leq 5)
    \end{align*}
        %just do it all you nerd
        let $d=0$ 
        \begin{align*}
          |0-5| \leq 2 \Rightarrow |0-2| \leq 5 &= 5 \leq 2 \Rightarrow 2\leq 5 \\
                                                &= False \Rightarrow True \\
                                                &= False
      \end{align*}
        

\end{enumerate}
 



\newpage
 \noindent $\square$ \textbf{Problem 2}  [\textsf{Logic and $\mathbb{N}$}] In \textsf{L0}, we encountered the infinite set of natural numbers $\mathbb{N} = \{0,1,2,3,4,5,6,7,8,9,10,11, 12, 13, \ldots\}$.  For each $k \in \mathbb{N}$, consider the truth set $$A_{k} = \{n \in \mathbb{N} \ : \ k \leq n\}$$
   
 
\noindent To prove $a \in A_k$, you have to verify $k \leq a$.  Prove each of the following propositions.  \begin{itemize}
  \itemsep0em 
  \item 2.1 [10 points] $\exists \ell \in \mathbb{N} \ \ell  \in A_7$
  \item 2.2 [10 points] $\exists \ell \in \mathbb{N} \ 7  \in A_{\ell}$
\item 2.3 [10 points] $\neg \big (| \mathbb{N} \setminus A_{10} | = 9 \big )$. \textit{Careful: if $S$ is a set, $|S|$ is the cardinality of $S$} 
\item 2.4 [10 points] $\forall m \in \mathbb{N} \ 2m+1 \in A_m$
\item 2.5 [10 points] $\forall m \in \mathbb{N} \ \big (m \in A_{112} \Rightarrow m \in A_{111} \big )$

 \end{itemize} 

\emph{solution}

        %solution goes here%
\begin{enumerate}[1)]
		\item  
			%sol'n% 
      
       $$ \exists \ell \in \mathbb{N} \ \ell  \in A_7 \\$$
       $$ A_7 = \{0,1,2,3,4,5,6,7\} $$
       let $\ell = 2$
       $$ 2 \in \mathbb{N} \land 2 \in A_7 $$

    

		\item
			%sol'n% 
     $$\exists \ell \in \mathbb{N} \ 7 \in A_{\ell}$$ 
      let $\ell = 7$ 
      $$7 \in \mathbb{N}$$
      $$7 \leq 7 \Rightarrow 7 \in A_7$$
      therefore: $7 \in \mathbb{N} \ 7 \in A_7$
      
		\item 

			%sol'n% 
      $$\neg \big (| \mathbb{N} \setminus A_{10} | = 9 \big )$$
      $$\mathbb{N} \setminus A_{10} = \{ 0,1,2,3,4,5,6,7,8,9,10 \} $$
      $$|\mathbb{N} \setminus A_{10}| = 11$$
      $$\neg \big( 10 = 9 \big )$$
      $$\neg False = True$$

		\item 
			%sol'n%		
      $$\forall m \in \mathbb{N} \ 2m+1 \in A_m$$
      $$A_m = \{1,2,3,...,m\}$$
      since $m +1 > m$, $m+1 \notin A_m$ b/c m is the largest element of $A_m$ 

		\item 
			%sol'n%  
      $$\forall m \in \mathbb{N} \ \big (m \in A_{112} \Rightarrow m \in A_{111} \big )$$
      $$A_{112} = \{1,2,3,...,111,112\}$$
       $$A_{111} = \{1,2,3,...,111\}$$
     since all elements of $A_{111}$ are in $A_{112}$ if an element is in $A_{111}$ it implies that that same element is also in $A_{112}$

	\end{enumerate}

\newpage
\noindent $\square$ \textbf{Bonus} [X points] Is $\forall n \in \mathbb{Z}_+ \ \Bigg ( \exists q \in \mathbb{Q} \ \Big( (0< q) \land (q <  \frac{1}{n} ) \Big ) \Bigg )$ true or false? Explain. \end{document}

\emph{solution}

        %solution goes here%
	



\end{document}


