\documentclass[11pt]{article}    

\usepackage{graphicx}
\usepackage{amsfonts}
\usepackage{amsmath}
\usepackage{amssymb}
\usepackage{amsmath,amscd}
\usepackage{amsthm}
\usepackage[T1]{fontenc}
\usepackage[hidelinks]{hyperref}
\usepackage{lmodern}
\hypersetup{
     colorlinks   = true,
     citecolor    = gray
}                      


%\usepackage{xypic}
\usepackage{mathrsfs}
\usepackage{upgreek}
\usepackage{esint}
\usepackage[usenames,dvipsnames,svgnames,table]{xcolor}
\usepackage[none]{hyphenat}
\setcounter{tocdepth}{3}
\usepackage{eucal} 
\usepackage{amsthm}
\usepackage{bbm}
\usepackage{textcomp}
\usepackage{wrapfig}
\renewcommand{\abstractname}{\vspace{-\baselineskip}}




\textwidth=6in
\textheight=9in
\hoffset=-0.375in
\voffset=-0.75in
\newtheorem{theorem}{Theorem}[subsection]
\newtheorem{corollary}{Corollary}[subsection]
\newtheorem{lemma}{Lemma}[subsection]
\newtheorem{remark}{Remark}
\newtheorem{claim}{Claim}
\newtheorem{proposition}{Proposition}[subsection]
\newtheorem{example}{Example}
\newtheorem{conjecture}{Conjecture}
\newtheorem{definition}{Definition}[subsection]
\newtheorem{problem}{Problem}[subsection]
\newtheorem{assumption}{Assumption}[subsection]
\newtheorem{condition}{Condition}[subsection]
\newtheorem{introdefinition}{Definition}
\newtheorem{alternatedefinition}{Alternate Definition}[subsection]


\numberwithin{equation}{section}






\makeatletter
\renewcommand\tableofcontents{%
    \@starttoc{toc}%
}
\makeatother


\newcommand*{\TitleFont}{%
      \usefont{\encodingdefault}{\rmdefault}{b}{n}%
      \fontsize{14}{14}%
      \selectfont}
\newcommand*{\AuthorFont}{%
      \usefont{\encodingdefault}{\rmdefault}{b}{n}%
      \fontsize{12}{12}%
      \selectfont}




\newcommand{\kbar}
{\mathchar'26\mkern-9mu k}

\newcommand{\ebar}
{\mathchar'26\mkern-9mu \theta_0}

\newcommand{\ebarblue}
{\mathchar'26\mkern-9mu \textcolor{blue}{\theta_0}}




\DeclareMathOperator*{\lcm}{lcm}
\DeclareMathOperator*{\R}{\mathbb{R}}
\DeclareMathOperator*{\N}{\mathbb{N}}
\DeclareMathOperator*{\Q}{\mathbb{Q}}
\DeclareMathOperator*{\Z}{\mathbb{Z}}
\DeclareMathOperator*{\E}{\mathbb{E}}
\DeclareMathOperator*{\C}{\mathbb{C}}
\DeclareMathOperator*{\A}{\mathbb{A}}
\DeclareMathOperator*{\expp}{\text{exp}}
\DeclareMathOperator*{\lt}{L_{\bullet}}
\DeclareMathOperator*{\roc}{\rotatebox[origin=c]{180}{c}}
\DeclareMathOperator*{\rok}{\rotatebox[origin=c]{180}{$k$}}
\DeclareMathOperator*{\boldv}{\boldsymbol{v}}
\DeclareMathOperator*{\hbarbrown}{\textcolor{brown}{\hbar}}
\DeclareMathOperator*{\e}{\textcolor{purple}{\theta_0_1}}
\DeclareMathOperator*{\ee}{\textcolor{purple}{\theta_0_2}}
\DeclareMathOperator*{\ered}{\textcolor{red}{--\theta_0_1 \theta_0_2}}
\DeclareMathOperator*{\eblue}{\textcolor{blue}{\theta_0_1 + \theta_0_2}}
\DeclareMathOperator*{\eebrown}{\textcolor{brown}{-- \theta_0}}
\DeclareMathOperator*{\ebrown}{\textcolor{brown}{\theta_0}}
\DeclareMathOperator*{\eebrownlevel}{\textcolor{brown}{\theta_0^2}}
\DeclareMathOperator*{\tealN}{\textit{\textcolor{teal}{N}}}
\DeclareMathOperator*{\bluenoise}{\boldsymbol{\Psi}_{\mathbb{T}}}
\DeclareMathOperator*{\pinknoise}{\boldsymbol{\Psi}_{\mathbb{T}}}
\DeclareMathOperator*{\uniform}{\boldsymbol{\rho}_{\star | \mathbb{T}}}
\DeclareMathOperator*{\vcurrent}{\widehat{\textit{\textbf{v}}}}
\newcommand{\dbar}
{\mathchar'26\mkern-9mu \theta_0}
 \setcounter{section}{0}

%my custom stuff
 \newtheorem{bprob}{$\square$ Problem}
 \usepackage{enumitem}
   %field symbols
   \DeclareMathOperator*{\OO}{\mathbb{O}}
   \DeclareMathOperator*{\II}{\mathbb{I}}
   \DeclareMathOperator*{\+}{\oplus}
   \DeclareMathOperator*{\x}{\otimes}
   \DeclareMathOperator*{\fl}{\prec} %fancy <
   \DeclareMathOperator*{\fg}{\succ} %fancy >
   \DeclareMathOperator*{\Lra}{\Leftrightarrow}
   \DeclareMathOperator*{\lra}{\leftrightarrow}
   %{enumerate}[label= 1.\arabic*, itemsep=0.2cm]

 
\begin{document}



\hypersetup{linkcolor=black}

\begin{center} 
\textbf{{MATH 112: \underline{Introduction to Analysis}}}\\
\textbf{{Fall 2024 Semester}}\\
\textbf{{Homework \textcolor{black}{5}: Due \textcolor{black}{Tuesday} \textcolor{black}{October} \textcolor{black}{14}, 10:00am PST}}\\
 \end{center}

 \iffalse
 \noindent \textbf{Instructions}
 \begin{itemize}
   \setlength\parskip{0pt}
 \itemsep0em
 \item Write your full name, ``Homework \textcolor{black}{5}'', and the date at the top of the first page.
 \item Show all work and explain your reasoning.  Write in complete sentences.
 \item Typeset your solutions in LaTeX.
  \item Each question has multiple parts. \textcolor{purple}{*\textit{New: restate each subquestion above your work}*}.
 \item \underline{Submit a single .pdf file to Gradescope under the assignment ``Homework \textcolor{black}{5}''.}
 \item \underline{You must use Gradescope to electronically match problems to pages in your .pdf}
 \item Questions? Email me or come to office hours.
 \item You are strongly encouraged to work together!  Just write up your own solutions.
 \end{itemize}
\fi 
 
 \medskip
 \noindent \textbf{Assignment} (2 Problems: 50 + 50 = 100 points total.) \medskip
 
   \noindent $\square$ \textbf{Problem 1}  [\textsf{Ordered Fields and Averages}] Let $(F, \preccurlyeq,  \oplus , \mathbb{O}, \otimes, \mathbb{I})$ be an ordered field, i.e. a set $F$ that is both a field $(F, \oplus, \mathbb{O}, \otimes, \mathbb{I})$ satisfying the 10 field axioms from \textsf{L7} and an ordered set $(F, \preccurlyeq)$ with a compatible total order $\preccurlyeq$ on $F$ satisfying the 4 ordered field axioms from \textsf{L11}. Prove each of the following propositions.  Indicate which axioms you use in each step.
   \begin{itemize}
  \itemsep0em 
   \item 1.1 [10 points] $\forall x \in F \ \forall y \in F \ \forall c  \in F \ \Big ( (x \prec y \ \land \ c \prec \mathbb{O}) \ \Rightarrow \ c \otimes x \succ c \otimes y \Big ) $\\
   \indent \ \ \ \ \  \ \ \ \ \ \ \ \  \ \ \ \ \ \ \ \textit{Note: recall from \textnormal{\textsf{L9}} that ``$a \succ b$'' means the same thing as ``$b \prec a$''.}
   \item 1.2 [10 points] $\forall x \in F \ (x \neq \mathbb{O} \ \Rightarrow \ \mathbb{O} \prec x^2 )$.  \textit{Recall from \textnormal{\textsf{L8}}: for all $x \in F$, $x^2 = x \otimes x$.}
 \item 1.3 [10 points] $- \mathbb{I} \prec \mathbb{O} \prec \mathbb{I}$.  \ \ \ \ \ \ \textit{Recall from \textnormal{\textsf{L7}}: $-x$ is the additive inverse of $x \in F$.}
      \item 1.4 [10 points] $\exists g \in F \ (\mathbb{I} \oplus \mathbb{I} ) \otimes g = \mathbb{I}$.
   \item 1.5 [10 points] $\forall x \in F \ \forall y \in F \ \Big ( (x \prec y) \Rightarrow (x \prec g \otimes ({x \oplus y}) \prec y) \Big )$ where the element\\    \indent \ \ \ \ \  \ \ \ \ \ \ \ \  \ \ \ \ \ \ \  $g \in F$ is the one you proved exists in the previous Problem 1.4.
   \end{itemize}
  %problem 1: 

   \begin{enumerate}[label= 1.\arabic*, itemsep=0.2cm]
     \item %1.1
       $\forall x \in F \ \forall y \in F \ \forall c  \in F \ \Big ( (x \fl y \ \land \ c \fl \OO) \ \Rightarrow \ c \x x \fg c \x y \Big ) $\\
       \iffalse 
       First: prove $(-a)\x b = - (a \x b)$.\\
       $-(a \x b)$ is additive inverse of $(a \x b)$. we know that by axiom 4, $-(a \x b) \+ (a \x b) = \OO$, 
       \begin{align*}
         ((-a) \x b) \+ (a \x b) &= b \x (-a \+ a)  \\ %# 4 
                                 &=  b \x \OO \\ %#9
                                 &= \OO %HW 4 1.2
       \end{align*}
        since both equal $\OO$, blah blah\\
      \fi
Assume $x \fl y \land c \fl \OO$. By the hypothetical strategy, if we can prove that \\ $c \x x \fg c \x y$, under this assumption, then the original statement is true. Start from our assumption.
       \begin{align*}
          &\Lra& x &\fl \ y \\ 
         &\Lra& (-c)\x x &\fl\ (-c) \x y \\%#4 & 14
          &\Lra& (c \x x) \+ ((-c) \x x) &\fl\ (c\x x) \+((-c)\x y) \\%#4 &13
          &\Lra& (c \+ (-c))\x x &\fl\ (c\x x) \+((-c)\x y) \\%#9
          &\Lra& \OO \x x &\fl\ (c\x x) \+((-c)\x y) \\%#4
        &\Lra&\OO &\fl\ (c\x c)\+((-c)\x y)\\%#Hw 4 1.2
        &\Lra& \OO \+ (c \x y) &\fl\ (c \x x) \+ ((-c) \x y))\+(c\x y) \\ %#13 
        &\Lra& \OO \+ (c\x y) &\fl\ (c \x x) \+ (y \x ((-c) \+ c)) \\ %#9
        &\Lra& \OO \+ (c\x y) &\fl\ (c \x x) \+ (y \x \OO) \\ %#4
        &\Lra& \OO \+ (c\x y) &\fl\ (c \x x) \+ \OO \\ %#by HW 4 p1.2 
           &\Lra& c \x y &\fl c \x x \\%#3 
        &\Lra& c \x x &\fg c \x y %L8 
       \end{align*}
       Using Axiom 14 we know that we can $\x$ any value to both sides, as long as it is greater than $\OO$. Since we have asusmed $c \fl \OO$, by axiom 4, $(-c) \fg \OO$ since $c \+ (-c) = \OO$. Next use axiom 14, which tells us we can $\+$ any element to both sides without affecting $\fl$ adding $(c \x x)$. Use Axiom 9, to distribute $\x$ over $\+$, on the left side which gets us $c \+ (-c) \x x)$. Then use axiom 4 to get that $(c \+ (-c))= \OO$. After this, use Hw 4, problem 1.2 to get $\OO \x x = \OO$. Use axiom $13$ to $\+ (c\x y)$ to  both sides. Use axiom 1 to group $(c \x x) \+ ((-c) \x y) \+ (x \x y)$ together, then use axiom 9 to distribute $\x$ over $\+$, to get: $y \x ((-c) \+ c)$. Use axiom 4, to get $(-c)\+ c) = \OO$. Use Hw 4, problem 1.2 to get $y \x \OO = \OO$. Use axiom 3, that any value plus $\OO$ equals itself to get: $(c \x x)$. Lastly use what we learned in L8, that you can flip $\fl$ to $\fg$ if you switch the terms sides as well. 
       
        
     \newpage
     \item %1.2
     $\forall x \in F \ (x \neq \mathbb{O} \ \Rightarrow \ \mathbb{O} \prec x^2 )$. \\
      EXPLANATION\\
      Assume $x \neq \OO$, means either $x < \OO$ or $x > \OO$.\\
      Case 1: $x < \OO$
      \begin{align*}
        &\Lra& x \fl& \OO \\ 
        &\Lra& x\x (-x) \fl&  \OO \x (-x) \\ %ax 4 and ax 4
        &\Lra& x \x (-x)  \fl& \OO \\ %by Hw 4 p 1.2
        &\Lra& (x \x x) \+ (x \x (-x)) \fl& (x \x x) \+ \OO \\%by ax13
        &\Lra& x \x (x \+ (-x)) \fl& (x \x x) \+ \OO \\ %by ax 9
        &\Lra& x \x \OO \fl& (x \x x) \+ \OO \\ %by ax 4
        &\Lra& \OO \fl& (x \x x) \+ \OO \\ %by Hw
        &\Lra& \OO \fl& (x \x x) \\ % by ax 3
        &\Lra& \OO \fl& x^2 %by L8
      \end{align*}
      EXPLANATION.\\
      Case 2: $x > \OO$
      \begin{align*}
        && x \fl& \OO \\
        &\Lra& x \x x \fl& \OO \x x \\ %by ax 4 since x is pos
        &\Lra& x \x x \fl& \OO \\ % vy hw 4 problem 1.2
        &\Lra& x^2 \OO \\%by L8
      \end{align*}
      EXPLANATION
      \smallskip\\
      Since this statement \_\_\_, holds for both possible cases and these the only possible cases that statisgy $x \neq \OO$, the orignal statment is True. 

     \item %1.3
      $- \mathbb{I} \prec \mathbb{O} \prec \mathbb{I}$.\\
      EXPLANATION\\
      by axiom 10, we have: $\OO \neq \II$ 
      \begin{align*}
       &&  \OO \neq& \II \\
       &&   \OO \fl& {\II}^2 \\ %by problem 1.2
       &\Lra& \OO \fl& \II \x \II \\%by L8
       &\Lra& \OO \fl& \II %by ax 7
      \end{align*}
      We now know that $\OO \fl \II$, use to show $-\II \fl \OO$.
      \begin{align*}
        && \OO \fl& \II \\
        &\Lra& (-\II)\+\OO \fl&(-\II) \+ \II \\ %by ax 13 & 4
        &\Lra& (-\II)\+\OO \fl& \OO \\%by ax 4
        &\Lra& -\II \fl& \OO %by ax 3
      \end{align*}


     \item %1.4
       $\exists g \in F \ (\mathbb{I} \oplus \mathbb{I} ) \otimes g = \mathbb{I}$.\\
      Let $g = (\II \+ \II)^{-1}$. 
      Substitute g into the equation.
      \begin{align*}
        (\II \+ \II) \x g &= (\II \x \II ) \x (\II \+ \II)^{-1}\\%by ax 8
                          &= \II \\%ax 8.
      \end{align*}
      EXPLANATION      
     \item %1.5
      $\forall x \in F \ \forall y \in F \ \Big ( (x \prec y) \Rightarrow (x \prec g \otimes ({x \oplus y}) \prec y) \Big )$\\
      Assume $x \fl y$, let $g = (\II \+ \II)^{-1}$.\\
      First show: $x \fl y \Rightarrow x \fl g \x (x \+ y )$
      \begin{align*}
        && x \fl& y \\
        &\Lra& x \+ x \fl& x \+ y \\ % by ax 13
        &\Lra& (x \x \II ) \+ (x \x \II) \fl& x \+ y \\ %by ax 7
        &\Lra& x \x (\II \+ ]II) \fl& x \+ y \\ %by ax 9
        &\Lra& (\II \+ \II) \x x \fl& x \+ y \\ %by ax 6
        &\Lra& (\II \+ \II) \x x \fl& (\II \+ \II)^{-1} ]x (x \+ y) \\ % by ax 8 and by ax 14, since \II > \OO, \II is pos, so (\II \x \II) is pos so (\II \x \II)^{-1} is pos.
        &\Lra& \II \x x \fl& (\II \x \II)^{-1} \x (x \x Y) \\ % by ax 14
        &\Lra& x \fl& (\II \x \II)^{-1} \x (x \x y) \\ % by ax 9
        &\Lra& x \fl& g \x (x \x y) %by choice of g
      \end{align*}
      EXPLANATION\\
Next show: $x \fl y \Rightarrow g \x (x \x y)$
\begin{align*}
  && x \fl& y \\
  &\Lra& x \+ y \fl& y \+ y \\ % by ax 13
  &\Lra& x \+ y \fl& (y \x \II)\+ (y \x \II) \\ % by ax 7
  &\Lra& x \+ y \fl& y \x (\II \+ \II) \\ % by ax 9
  &\Lra& (x \+ y) \x (\II \+ \II)^{-1} \fl& y \x (\II \+ \II) \x (\II \x \II)^{-1} \\ % by ax 7
  &\Lra& (x \+ y) \x (\II \+ \II)^{-1} \fl& y \x ((\II \+ \II) \x (\II \x \II)^{-1}) \\ % by ax ?
  &\Lra& (x \+ y) \x (\II \+ \II)^{-1} \fl& y \x \II \\ % by ax 8
  &\Lra& (x \+ y) \x (\II \+ \II)^{-1} \fl& y  \\ % by ax 7
\end{align*}
     EXPLANATION\\ 
MORE EXPLANATION Since we have $a\fl b \land b \fl c$. by axiom 11, this implies we have $a \fl c$. furthermore we have $a\fl b\fl c$.



   \end{enumerate}


\newpage
    \noindent $\square$ \textbf{Problem 2}  [\textsf{Optimal Bounds of Intervals in the Continuum}] Recall from \textsf{L9} that in any totally ordered set $(S, \preceq)$, if $A \subseteq S$ is a subset, we say that $u_{\star} = \sup S$ is an optimal upper bound of $A$ in $S$ if (i) $u_{\star}$ is an upper bound for $A$ in $(S, \preceq)$ and (ii) any upper bound $u$ of $A$ satisfies $u_{\star} \preceq u$.  In \textsf{L9}, we saw that special subsets of $S$ are the closed and open intervals $$[a,b]_S := \Big \{ x \in S \ : \ a \preceq x \preceq b \Big \}$$
    $$(a,b)_S := \Big \{ x \in S \ : \ a \prec x \prec b \Big \}.$$
    
              \begin{itemize} 
   \itemsep0em 
 \item 2.1 [25 points] \ Prove that $\sup [111,112]_{\mathbb{R}} = 112$.
   
 \item 2.2 [25 points] \ Prove that $\sup (111,112)_{\mathbb{R}} = 112$.
\end{itemize}

  \begin{enumerate}[label= 2.\arabic*, itemsep=0.2cm]
    \item %2.1
      Prove that $\sup [111,112]_{\mathbb{R}} = 112$.\\
      We are given, in l9 and in the begining of this problem, that:  
      $$[a,b]_S := \Big \{ x \in S \ : \ a \preceq x \preceq b \Big \}.$$ 
      Thus: 
      $$[112,112]_{\R} := \Big \{ x\in \R :\ 112 \preceq x \preceq 112 \Big \}.$$ 
      To prove that $\sup [111,112]_{\mathbb{R}} = 112$, we will first show that 112 is an upper bound for $[111,112]_{\mathbb{R}}$.
      Recall that, from L9, the definition: an upper bound for $A$ in $S$ is an element $u \in S$ so $\forall \alpha \in A \ \alpha \leq u$. 
      To prove that 112 is an upper bound, it is sufficent to show that $\forall x \in \R \ x \preceq 112$.
      
    \item %2.2
      Prove that $\sup (111,112)_{\mathbb{R}} = 112$.\\
      We are given, in l9 and in the begining of this problem, that:  
      $$(a,b)_S := \Big \{ x \in S \ : \ a \fl x \fl b \Big \}.$$ 
      Thus: 
      $$(112,112)_{\R} := \Big \{ x\in \R :\ 111 \fl x  \fl 112\Big \}.$$ 
      To prove that $\sup [111,112]_{\mathbb{R}} = 112$, we will first show that 112 is an upper bound for $[111,112]_{\mathbb{R}}$.
      Recall that, from L9, the definition: an upper bound for $A$ in $S$ is an element $u \in S$ so $\forall \alpha \in A \ \alpha \leq u$. 
      To prove that 112 is an upper bound, it is sufficent to show that $\forall x \in \R \ x \preceq 112$.
      

   \end{enumerate}


  

    
\noindent $\square$ \textbf{Bonus} [X points] Despite our proofs in \textsf{L6} that the set of rational numbers $\mathbb{Q}$ is countably infinite and the set of irrational numbers $\mathbb{R} \setminus \mathbb{Q}$ is uncountably infinite, prove that\\
\indent \ (i) between any two rational numbers there is an irrational number\\
\indent (ii) between any two irrational numbers there is a rational number.




 \end{document}
